\documentclass[10pt,aspectratio=169,dvipsnames]{beamer}
\usetheme[]{Berlin}

\setbeamertemplate{footline}{
  \usebeamercolor[fg]{framesource}%
  \usebeamerfont{page number in head}%
  \hspace{0.2cm}
  \small \insertframenumber
  \vspace{0.2cm}
  \doclicenseIcon
  \hfill
  \hspace{0.2cm}
}

\setbeamercovered{transparent}

\setbeamertemplate{footline}[
 myframe number]

% PACKAGES
\usepackage[absolute,overlay]{textpos}
\usepackage[utf8]{inputenc}
\usepackage[official]{eurosym}
\usepackage{booktabs}
\usepackage{parskip}
\usepackage{bm}
\usepackage{tikz}
\usepackage{adjustbox}
\usepackage{tabularx}
\usepackage{gensymb}

\usepackage[
    type={CC},
    modifier={by},
    version={4.0},
]{doclicense}

% HYPERREFERENCES
\usepackage{hyperref}
\hypersetup{
	colorlinks=true,
	citecolor=tub-blue,
	linkcolor=tub-blue,
	urlcolor=tub-blue
}

% GRAPHICS	
\graphicspath{{../graphics/}}
\DeclareGraphicsExtensions{.pdf,.jpeg,.png,.jpg}

% FORMATTING
\setlength{\parskip}{6pt}
\linespread{1.1}
\newcommand{\seprule}{\par\noindent\textcolor{black!25}{\rule{\textwidth}{0.4pt}}}

% SOURCES
\setbeamercolor{framesource}{fg=gray}
\setbeamerfont{framesource}{size=\tiny}
\newcommand{\source}[1]{\begin{textblock*}{5cm}(8.6cm,8.25cm)
    \begin{beamercolorbox}[ht=0.5cm,right]{framesource}
        \usebeamerfont{framesource}\usebeamercolor[fg]{framesource} Source: {#1}
    \end{beamercolorbox}
\end{textblock*}}

% COLOR TEXT
\newcommand{\bl}[1]{\textcolor{tub-blue}{#1}}
\newcommand{\gr}[1]{\textcolor{tub-green}{#1}}
\newcommand{\rd}[1]{\textcolor{tub-red}{#1}}
\newcommand{\yl}[1]{\textcolor{tub-yellow}{#1}}

\usepackage{array}% http://ctan.org/pkg/array
\makeatletter
\g@addto@macro{\endtabular}{\rowfont{}}% Clear row font
\makeatother
\newcommand{\rowfonttype}{}% Current row font
\newcommand{\rowfont}[1]{% Set current row font
   \gdef\rowfonttype{#1}#1%
}
\newcolumntype{L}{>{\rowfonttype}l}
\newcolumntype{R}{>{\rowfonttype}r}

% FONT
% \usepackage{utopia} 

% REFERENCES
%\usepackage[backend=biber,style=authoryear-comp]{biblatex}
%\bibliography{references.bib}

% TITLE PAGE
\title{Introduction to the Seminars}

\author{\bf Prof. Dr. Tom Brown, Dr. Iegor Riepin, Philipp Glaum}

\institute[Technische Universität Berlin] % (optional, but mostly needed)
{
  \normalsize
  Technische Universität Berlin\\
  Department of Digital Transformation in Energy Systems \\
  Institute of Energy Technology
}

\date{22 November 2023}

\subject{Modelling the European Energy System}

\titlegraphic{%
	\includegraphics[trim=0 0cm 0 0cm,height=1.7cm,clip=true]{../graphics/tublogo.pdf}
}

\begin{document}

{
\setbeamertemplate{footline}{}
\maketitle
}
\addtocounter{framenumber}{-1}

\begin{frame}
  \frametitle{Outline}
  \setbeamertemplate{section in toc}[sections numbered]
  \tableofcontents[hideallsubsections]
\end{frame}


\section{General Information}

\begin{frame}{Overview}

  \begin{itemize}
    \item Seminars:
          \begin{itemize}
            \item New Research in Energy System Modelling:
                  \\ \href{https://isis.tu-berlin.de/course/view.php?id=36589}{https://isis.tu-berlin.de/course/view.php?id=36589}
            \item New Development in Energy Markets:\\ \href{https://isis.tu-berlin.de/course/view.php?id=35945}{https://isis.tu-berlin.de/course/view.php?id=35945}
          \end{itemize}
    \item Credit points: 3 ECTS
    \item Language: English/German
  \end{itemize}

\end{frame}

\begin{frame}{Expectations}

  \begin{itemize}
    \item Day of presentations: 20-min presentation of a selected topic, 15-min discussion, 5-min feedback from supervisors and students
    \item Several consultations with your supervisor
    \begin{itemize}
        \item Better to start early -- recommended first meeting after topic allocation
        \item Better to be prepared -- you can have a valuable discussion with your supervisor
        \item At least one meeting with your supervisor is mandatory
    \end{itemize}
    \item Attend all seminar presentations \&  participate in discussions
    \item Evaluation criteria are uploaded to ISIS
  \end{itemize}

\end{frame}


\begin{frame}{Organisational Classification}

    \begin{block}{Seminar only}
          \begin{tabularx}{0.8\textwidth}{l l l}
            \textbf{Free choice / Freie Wahl} & & \textbf{3 ECTS}\\
            Seminar/Vortragsreihe & Presentation,participation & 3 ECTS
          \end{tabularx}
    \end{block}
    \begin{block}{Portfolio examination}
    \begin{tabularx}{0.8\textwidth}{l l l}
            \textbf{Energy Systems (Project EVT)} & & \textbf{9 ECTS}\\
            Energy Systems (SS)   & Lecture, tutorial, written exam & 6 ECTS \\
            Seminar/Vortragsreihe & Presentation, participation     & 3 ECTS
      \end{tabularx}
      \newline
      \newline
      Eligible study programs: Master EVT, RES, Industrial Engineering and Management
  \end{block}
          


\end{frame}


\begin{frame}{Timeline}

  \begin{itemize}
    \item Binding exam registration until \textbf{26 November 2023} through the MTS platform.
    \item For registered students we open "Topic allocation" polls on ISIS platform: 
        \begin{itemize}
            \item Allocation algorithm considers your preferences and maximises the overall happiness
            \item \textbf{not} "First come, first served"
          \end{itemize}
    \item Limited spots: students with seminar as a part of 9CP course are prioritized
    \item Presentation dates:
          \begin{itemize}
            \item New Research in Energy System Modelling: \textbf{18 March 2024}
            \item New Developments in Energy Markets: \textbf{21 \& 22  March 2024}
          \end{itemize}
  \end{itemize}

\end{frame}


% \begin{frame}{Time schedule for registration}

%   \begin{itemize}
%     \item Registration via questionnaire (25 May - 2 June)
%     \item Topic allocation (25th of May - 2nd June)
%     \item Announcement of topic allocation (12th June)
%     \item Registration on MTS (12th to 19th June)
    
%   \end{itemize}

% \end{frame}

%%%%%%%%%%%%%%%%%%%%%%%%%%%%%%%%%%%%%%%%%%%%%%%%%%%%%%%%%%%%

\section{Topics: New Research in Energy System Modelling}

\begin{frame}
Seminar topics for {\bf "New Research in Energy System Modelling"}

You will be supervised by the research staff of the two departments:  
    \begin{itemize}
        \item Department of \href{https://www.ensys.tu-berlin.de/menue/overview/}{Digital Transformation in Energy Systems} at TU Berlin lead by Prof.~Tom~Brown
        \item Department of \href{https://www.pik-potsdam.de/en/institute/departments/transformation-pathways/research/energy-systems}{Energy Systems} at Potsdam Institute for Climate Impact Research lead by Prof.~Gunnar~Luderer.
    \end{itemize}
\end{frame}


\begin{frame}
  \begin{block}{Topic 1: Flexible green hydrogen: The effect of relaxing simultaneity requirements on project design, economics, and power sector emissions}
      
    Ruhnau \& Schiele (2023), \href{https://doi.org/10.1016/j.enpol.2023.113763}{https://doi.org/10.1016/j.enpol.2023.113763}
    
    \enquote{In many net-zero energy scenarios, electrolytic hydrogen is a key component to decarbonize hard-to-abate sectors and to provide flexibility to the power sector. In current energy systems that are not yet fully decarbonized, however, the hydrogen ramp-up raises the concern of increasing power sector emissions. To avoid such additional emissions, recent EU regulation defines requirements for electrolytic hydrogen to qualify as green along three dimensions: the additionality, the proximity, and the simultaneity of renewable electricity generation. Focusing on the temporal dimension, this article investigates the effects of a strict hourly simultaneity requirement, full temporal flexibility, as well as simultaneity exemptions in the current EU regulation. [..] We argue that current energy transition trends further lower the risk of increasing power sector emissions under a flexible definition of green hydrogen and recommend this as the way forward for a sustainable hydrogen policy.} 
  
    \hfill
    Topic supervisor: Elisabeth Zeyen, e.zeyen@tu-berlin.de
    
  \end{block}
\end{frame}


\begin{frame}
  \begin{block}{Topic 2: A global model of hourly space heating and cooling demand at multiple spatial scales}
      
    Staffel et al. (2023), \href{https://www.nature.com/articles/s41560-023-01341-5}{https://www.nature.com/articles/s41560-023-01341-5}
    
    \enquote{Accurate modelling of the weather's temporal and spatial impacts on building energy demand is critical to decarbonizing energy systems. Here we introduce a customizable model for hourly heating and cooling demand applicable globally at all spatial scales. We validate against demand from ~5,000 buildings and 43 regions across four continents. The model requires limited data inputs and shows better agreement with measured demand than existing models. We use it first to demonstrate that a 1\degree\hspace{0pt}C reduction in thermostat settings across all buildings could reduce Europe's gas consumption by 240~TWh~yr$^{-1}$, approximately one-sixth of historical imports from Russia. Second, we show that service demand for cooling is increasing by up to 5\% per year in some regions due to climate change, and 5 billion people experience >100 additional cooling degree days per year when compared with a generation ago. The model and underlying data are freely accessible to promote further research.}

    \hfill
    Topic supervisor: Iegor Riepin, iegor.riepin@tu-berlin.de
    
  \end{block}
\end{frame}


\begin{frame}
  \begin{block}{Topic 3: Opportunities for flexible electricity loads such as hydrogen production from curtailed generation}
      
    Ruggles et al. (2021), \href{https://doi.org/10.1016/j.adapen.2021.100051}{https://doi.org/10.1016/j.adapen.2021.100051}
    
    \enquote{Variable, low-cost, low-carbon electricity that would otherwise be curtailed may provide a substantial economic opportunity for entities that can flexibly adapt their electricity consumption. We used historical hourly weather data over the contiguous U.S. to model the characteristics of least-cost electricity systems dominated by variable renewable generation that powered firm and flexible electricity demands (loads). [..] The macro-scale energy model indicated that variable renewable electricity systems optimized to supply firm loads at current costs could supply 25\% or more additional flexible load with minimal capacity expansion, while resulting in reduced average electricity costs (10\% or less capacity expansion and 10\% to 20\% reduction in costs in our modeled scenarios). These results indicate that adding flexible loads to electricity systems will likely allow more full utilization of generation assets across a wide range of system architectures, thus providing new energy services with infrastructure that is already needed to supply firm electricity loads.}

    \hfill
    Topic supervisor: Philipp Glaum, p.glaum@tu-berlin.de
    
  \end{block}
\end{frame}


\begin{frame}
  \begin{block}{Topic 4: Global fossil fuel reduction pathways under different climate mitigation strategies and ambitions}
      
    Achakulwisut et al. (2023), \href{https://www.nature.com/articles/s41467-023-41105-z}{https://www.nature.com/articles/s41467-023-41105-z}
    
    \enquote{The mitigation scenarios database of the Intergovernmental Panel on Climate Change's Sixth Assessment Report is an important resource for informing policymaking on energy transitions. However, there is a large variety of models, scenario designs, and resulting outputs. Here we analyse the scenarios consistent with limiting warming to 2\degree\hspace{0pt}C or below regarding the speed, trajectory, and feasibility of different fossil fuel reduction pathways. In scenarios limiting warming to 1.5\degree\hspace{0pt}C with no or limited overshoot, global coal, oil, and natural gas supply (intended for all uses) decline on average by 95\%, 62\%, and 42\%, respectively, from 2020 to 2050, but the long-term role of gas is highly variable. Higher-gas pathways are enabled by higher carbon capture and storage (CCS) and carbon dioxide removal (CDR), but are likely associated with inadequate model representation of regional CO$_2$ storage capacity and technology adoption, diffusion, and path-dependencies. If CDR is constrained by limits derived from expert consensus, the respective modelled coal, oil, and gas reductions become 99\%, 70\%, and 84\%. Our findings suggest the need to adopt unambiguous near- and long-term reduction benchmarks in coal, oil, and gas production and use alongside other climate mitigation targets.}

    \hfill
    Topic supervisor: Philipp Verpoort, philipp.verpoort@pik-potsdam.de
    
  \end{block}
\end{frame}


\begin{frame}
  \begin{block}{Topic 5: Role of hydrogen-based energy carriers as an alternative option to reduce residual emissions associated with mid-century decarbonization goals}
      
    Oshiro \& Fujimori (2022), \href{https://doi.org/10.1016/j.apenergy.2022.118803}{https://doi.org/10.1016/j.apenergy.2022.118803}
    
    \enquote{Hydrogen-based energy carriers, including hydrogen, ammonia and synthetic hydrocarbons, are expected to help reduce residual carbon dioxide emissions in the context of the Paris Agreement goals, although their potential has not yet been fully clarified in light of their competitiveness and complementarity with other mitigation options such as electricity, biofuels and carbon capture and storage (CCS). This study aimed to explore the role of hydrogen in the global energy system under various mitigation scenarios and technology portfolios using a detailed energy system model that considers various energy technologies including the conversion and use of hydrogen-based energy carriers. The results indicate that the share of hydrogen-based energy carriers generally remains less than 5\% of global final energy demand by 2050 in the 2\degree\hspace{0pt}C scenarios. Nevertheless, such carriers contribute to removal of residual emissions from the industry and transport sectors under specific conditions. Their share increases to 10-15\% under stringent mitigation scenarios corresponding to 1.5\degree\hspace{0pt}C warming and scenarios without CCS. The transport sector is the largest consumer, accounting for half or more of hydrogen production, followed by the industry and power sectors. [..]}

    \hfill
    Topic supervisor: Adrian Odenweller, adrian.odenweller@pik-potsdam.de
    
  \end{block}
\end{frame}


\begin{frame}
  \begin{block}{Topic 6: Electricity pricing challenges in future renewables-dominant power systems}
      
    Mallapragada et al. (2023), \href{https://doi.org/10.1016/j.eneco.2023.106981}{https://doi.org/10.1016/j.eneco.2023.106981}
    
    \enquote{[..] We have modeled optimized, deeply decarbonized power systems in three US regions at mid-century under a wide range of plausible cost and technology assumptions. The shadow marginal values of energy (MVEs) from these optimizations approximate the wholesale spot prices of energy in simplified hypothetical competitive energy-only wholesale markets in which revenues earned by selling energy in wholesale electricity markets should be sufficient to cover all capital and operating costs. A very robust result is that under carbon constraints, very low MVEs occur much more frequently than in today's wholesale markets, and very high prices are also more frequent than today. Revenues from a relatively small number of high MVE periods are required to fully cover VRE capital and operating costs, while storage charges and discharges in many hours. In an ideal, efficient regime, a competitive energy-only wholesale market without price caps would minimize total system costs, and retail rates equal to wholesale spot prices would fully cover those costs and induce efficient demand behavior. Real power systems depart significantly from this ideal: price caps are used frequently in wholesale markets, capacity payments from organized capacity markets or bilateral contracts are relied on to supplement energy market revenues to enable full cost recovery, [..]}

    \hfill
    Topic supervisor: Fabian Neumann, f.neumann@tu-berlin.de
    
  \end{block}
\end{frame}


\begin{frame}
  \begin{block}{Topic 7: Cost and Efficiency Requirements for Successful Electricity Storage in a Highly Renewable European Energy System}
      
    Gøtske et al. (2023), \href{https://doi.org/10.1103/PRXEnergy.2.023006}{https://doi.org/10.1103/PRXEnergy.2.023006}
    
    \enquote{Future highly renewable energy systems might require substantial storage deployment. At the current stage, the technology portfolio of dominant storage options is limited to pumped-hydro storage and Li-ion batteries. It is uncertain which storage design will be able to compete with these options. Considering Europe as a case study, we derive the cost and efficiency requirements of a generic storage technology, which we refer to as \emph{storage-X}, to be deployed in the cost-optimal system. [..] Based on a sample space of 724 storage configurations, we show that energy capacity cost and discharge efficiency largely determine the optimal storage deployment, in agreement with previous studies. Here, we show that charge capacity cost is also important due to its impact on renewable curtailment. A significant deployment of storage-X in a cost-optimal system requires (a) discharge efficiency of at least 95\%, (b) discharge efficiency of at least 50\% together with low energy capacity cost (10~€/kWh), or (c) discharge efficiency of at least 25\% with very low energy capacity cost (2~€/kWh). Comparing our findings with seven emerging technologies reveals that none of them fulfill these requirements.  [..]}

    \hfill
    Topic supervisor: Christoph Tries, christoph.tries@tu-berlin.de
    
  \end{block}
\end{frame}


\begin{frame}
  \begin{block}{Topic 8: Identifying optimal technological portfolios for European power generation towards climate change mitigation: A robust portfolio analysis approach}
      
    Forouli et al. (2019), \href{https://doi.org/10.1016/j.jup.2019.01.006}{https://doi.org/10.1016/j.jup.2019.01.006}
    
    \enquote{Here, an integrative approach is proposed to link integrated assessment modelling results from the GCAM model with a novel portfolio analysis framework. This framework comprises a bi-objective optimisation model, Monte Carlo analysis and the Iterative Trichotomic Approach, aimed at carrying out stochastic uncertainty assessment and enhancing robustness. The approach is applied for identifying optimal technological portfolios for power generation in the EU towards climate change mitigation until 2050. The considered technologies include photovoltaics, concentrated solar power, wind, nuclear, biomass and carbon capture and storage, for which different subsidy curves for emissions reduction and energy security are considered.}

    \hfill
    Topic supervisor: Chris Gong Chen.Gong@pik-potsdam.de
    
  \end{block}
\end{frame}


\begin{frame}
  \begin{block}{Topic 9: Climate change mitigation potential of carbon capture and utilization in the chemical industry}
      
    Kätelhön et al. (2019), \href{https://doi.org/10.1073/pnas.1821029116}{https://doi.org/10.1073/pnas.1821029116}
    
    \enquote{Chemical production is set to become the single largest driver of global oil consumption by 2030. To reduce oil consumption and resulting greenhouse gas (GHG) emissions, carbon dioxide can be captured from stacks or air and utilized as alternative carbon source for chemicals. Here, we show that carbon capture and utilization (CCU) has the technical potential to decouple chemical production from fossil resources, reducing annual GHG emissions by up to 3.5 Gt CO$_2$-eq in 2030. Exploiting this potential, however, requires more than 18.1 PWh of low-carbon electricity, corresponding to 55\% of the projected global electricity production in 2030. Most large-scale CCU technologies are found to be less efficient in reducing GHG emissions per unit low-carbon electricity when benchmarked to power-to-X efficiencies reported for other large-scale applications including electro-mobility (e-mobility) and heat pumps. Once and where these other demands are satisfied, CCU in the chemical industry could efficiently contribute to climate change mitigation.}

    \hfill
    Topic supervisor: Tom Brown, t.brown@tu-berlin.de
    
  \end{block}
\end{frame}


\begin{frame}
  \begin{block}{Topic 10: Committed emissions from existing energy infrastructure jeopardize 1.5\degree\hspace{0pt}C climate target}
      
    Tong et al. (2019), \href{https://doi.org/10.1038/s41586-019-1364-3}{https://doi.org/10.1038/s41586-019-1364-3}
    
    \enquote{Net anthropogenic emissions of carbon dioxide (CO$_2$) must approach zero by mid-century (2050) in order to stabilize the global mean temperature at the level targeted by international efforts. Yet continued expansion of fossil-fuel-burning energy infrastructure implies already ‘committed’ future CO$_2$ emissions. Here we use detailed datasets of existing fossil-fuel energy infrastructure in 2018 to estimate regional and sectoral patterns of committed CO$_2$ emissions, the sensitivity of such emissions to assumed operating lifetimes and schedules, and the economic value of the associated infrastructure. [...] Our estimates suggest that little or no new CO$_2$-emitting infrastructure can be commissioned, and that existing infrastructure may need to be retired early (or be retrofitted with carbon capture and storage technology) in order to meet the Paris Agreement climate goals. Given the asset value per tonne of committed emissions, we suggest that the most cost-effective premature infrastructure retirements will be in the electricity and industry sectors, if non-emitting alternatives are available and affordable.}

    \hfill
    Topic supervisor: Gunnar Luderer, luderer@pik-potsdam.de
    
  \end{block}
\end{frame}



\begin{frame}
  \begin{block}{Topic 11: Delaying carbon dioxide removal in the European Union puts climate targets at risk}
      
    Galán-Martín et al. (2021), \href{https://www.nature.com/articles/s41467-021-26680-3}{https://www.nature.com/articles/s41467-021-26680-3}
    
    \enquote{Carbon dioxide removal (CDR) will be essential to meet the climate targets, so enabling its deployment at the right time will be decisive. Here, we investigate the still poorly understood implications of delaying CDR actions, focusing on integrating direct air capture and bioenergy with carbon capture and storage (DACCS and BECCS) into the European Union power mix. Under an indicative target of 50 Gt of net CO$_2$ by 2100, delayed CDR would cost an extra of 0.12-0.19 trillion EUR per year of inaction. Moreover, postponing CDR beyond mid-century would substantially reduce the removal potential to almost half (35.60 Gt CO$_2$) due to the underused biomass and land resources and the maximum technology diffusion speed. The effective design of BECCS and DACCS systems calls for long-term planning starting from now and aligned with the evolving power systems. Our quantitative analysis of the consequences of inaction on CDR--with climate targets at risk and fair CDR contributions at stake—should help to break the current impasse and incentivize early actions worldwide.}

    \hfill
    Topic supervisor:  Anne Merfort, amerfort@pik-potdam.de
    
  \end{block}
\end{frame}


%%%%%%%%%%%%%%%%%%%%%%%%%%%%%%%%%%%%%%%%%%%%%%%%%%%%%%%%%%%%

\section{Topics: New Development in Energy Market}

\begin{frame}
Seminar topics for {\bf "New Developments in Energy Markets"}

You will be suppervised by:  
    \begin{itemize}
        \item  Research staff of the department of \href{https://www.ensys.tu-berlin.de/menue/overview/}{Digital Transformation in Energy Systems} at TU Berlin lead by Prof.~Tom~Brown
        \item Prof. Georg Erdmann
        \item Prof. Andreas Grübel
    \end{itemize}
\end{frame}


\begin{frame}
  \begin{block}{Markets for carbon dioxide removal}
    While you can purchase certificates for carbon emissions, there are no current schemes that offer compensation for removal of carbon dioxide from the air (CDR). Some suggest that emissions and removal markets should be developed separately to encourage investment in CDR, so that the markets can later be coupled. Some have proposed intertemporal “carbon debts”.\\
    \href{https://www.frontiersin.org/articles/10.3389/fclim.2021.690023/full}{Integrating Carbon Dioxide Removal Into European Emissions Trading}\\
    \href{https://www.swp-berlin.org/publications/products/research_papers/2020RP08_ClimateMitigation.pdf}{Research Paper on Unconventional Mitigation}\\
    \href{https://www.nature.com/articles/s41586-021-03723-9}{Operationalizing the net-negative carbon economy}\\
    \href{https://blogs.microsoft.com/blog/2021/01/28/one-year-later-the-path-to-carbon-negative-a-progress-report-on-our-climate-moonshot/}{Microsoft's path to "climate negative"}\\
    \href{https://www.ft.com/content/69c5e964-a91a-42b8-818d-6a5d9b21b6cd}{UK emission trading scheme plans to adopt credits for direct air capture}\\
    \href{https://www.theatlantic.com/science/archive/2022/04/big-tech-investment-carbon-removal/629545/}{Big Tech invests \$900 million in CDR}
    \end{block}  
\end{frame}

\begin{frame}
    \begin{block}{ETS 2: Heating and Transport}
    In 2023 the European Union signed off on a new parallel ETS for transport and buildings to start in 2027/8. You will describe its features and differences to the ETS 1 for energy and industry.\\
    \href{https://energypost.eu/understanding-the-new-eu-ets-part-2-buildings-road-transport-fuels-and-how-the-revenues-will-be-spent/}{Link 1}
    \end{block}
    \begin{block}{Synthetic Inertia markets in Great Britain}
    At even shorter time scales than balancing power acts, the inertia of rotating synchronous generators helps to stop the frequency changing too fast. But what happens when most of the generation comes from PV and wind inverters that don’t have any inherent inertia? Inverters can be adapted to provide synthetic inertia, or generators can be run without a prime mover in synchronous condenser mode. Great Britain has led the way in new markets for synthetic inertia. You will explain what synthetic inertia is, how the market is structured, what players are on the market and how well it is working.\\
    \href{https://www.nationalgrideso.com/news/green-inertia-projects-and-world-first-tech-tell-british-energy-success-story}{Link 1}
    \href{https://arxiv.org/abs/2208.04869}{Link 2}
    \end{block}
\end{frame}


\begin{frame}
    \begin{block}{Flexibilisierung von Biogasanlagen}
    In der Vergangenheit liefen viele Biogasanlagen in Grundlastbetrieb. Der Gesetzgeber versucht Anreize für einen marktorientierten flexiblen Betrieb zu schaffen. Sie liefern einen Überblick des Status-Quos und wie die Flexibilisierung mit der Maximierung von Biogas-Produktion für die Gaskrise zusammenpasst.\\
    \href{https://www.kwk-flexperten.net/flexibilisierung-von-biogasanlagen}{Link 1}
    \href{https://biogas.fnr.de/biogas-nutzung/stromerzeugung/flexibilisierung-von-biogasanlagen}{Link 2}
    \href{https://www.dbfz.de/fileadmin//user_upload/Referenzen/Studien/160323_Externes_Hintergrundpapier_Flexibilisierung_von_Biogasanlagen_in_Deutschland.pdf}{Link 3}
    \end{block}
    \begin{block}{Präqualifikation Offshore-Wind für Reserveleistung}
    In Mai 2022 hat Offshore-Wind zum ersten Mal für Regelleistung in Form von Minutenreserve und Sekundärreserve präqualifiziert. Sie erklären, worum es geht und welche Herausforderungen im Weg standen.\\
    \href{https://orsted.de/presse-media/news/2022/05/borkum-riffgrund-1-regelenergie}{Link 1}
    \end{block}
    \begin{block}{Capacity markets in France}
    For some years France has run capacity markets for power generation. You will describe how the market works and what lessons there might be for other countries like Germany.\\
    \href{https://www.eex.com/en/services/registry-services/french-capacity-guarantees-for-rte}{Link 1}
    \href{https://www.services-rte.com/en/learn-more-about-our-services/participate-in-the-capacity-mechanism.html}{Link 2}
    \href{https://www.agora-energiewende.de/en/projects/comparing-capacity-market-designs-in-france-and-potentially-germany/}{Link 3}
    \end{block}
\end{frame}
\begin{frame}
    \begin{block}{Australian electricity market suspension in June 2022}
    In June 2022 the Australian National Electricity Market (NEM) was suspended for several days due to irregularities. Difficult conditions included a confluence of high commodity prices, domestic market price caps, planned and unplanned outages of scheduled generating plant, low output from semi-scheduled generation, and high winter demand conditions. You will explain what happened and what lessons can be learned.\\
    \href{https://www.aemo.com.au/-/media/files/electricity/nem/market_notices_and_events/market_event_reports/2022/nem-market-suspension-and-operational-challenges-in-june-2022.pdf}{Link 1}
    \end{block}
    \begin{block}{Gesetzes zum Neustart der Digitalisierung der Energiewende}
    Zentrales Ziel des Gesetzes zum Neustart der Digitalisierung der Energiewende ist es, den Smart-Meter-Rollout zu beschleunigen. Warum gibt es Verzögerungen und wie sollen die Änderungen nun wirken?\\
    \href{https://dserver.bundestag.de/btd/20/064/2006457.pdf}{Link 1}\\
    \href{https://www.gesetze-im-internet.de/messbg/}{Stute, Judith, and Matthias Kühnbach. "Dynamische Stromtarife unter Berücksichtigung des Nutzendenverhaltens: Auswirkungen auf das Verteilnetz." 12. Internationale Energiewirtschaftstagung an der TU Wien (2021).}
    \end{block}
\end{frame}

\begin{frame}
    \begin{block}{Dynamische Tarife}
    Das Gesetz zum Neustart der Digitalisierung der Energiewende soll die Einführung dynamischer Tarife ermöglichen. Für die Energiewende ist es unabdingbar, auf der Verbraucherseite dynamische und flexible Tarife und Demand Side Management flächendeckend umzusetzen. Welche Produkte und Lösungen gibt es?\\
    \href{https://www.agora-energiewende.de/fileadmin/Projekte/2017/Abgaben_Umlagen/146_Neue-Preismodelle_WEB.pdf}{Link 1}
    \href{https://tibber.com/de}{Link 2}
    \end{block}
    \begin{block}{Portfolio- und Risikomanagement in Krisenzeiten}
    Energieversorgungsunternehmen (EVU) standen in den vergangenen Monaten vor großen Herausforderungen: Zum einen mussten sie Kunden aufnehmen, weil Konkurrenten die Belieferung eingestellt haben, zum anderen konnten sie diesen Kunden keine Marktpreise anbieten. Darüber hinaus fielen Vorlieferanten der EVU aus und die verbleibenden Lieferanten verlangten plötzlich Sicherheiten. Analysieren Sie die aktuelle Situation und wagen Sie einen Ausblick, wie sich die EVU-”Landschaft” verändern wird.\\
    \href{https://www.bafin.de/SharedDocs/Downloads/DE/Rundschreiben/dl_rs1021_MaRisk_Erlaeuterungen.pdf?__blob=publicationFile&v=5}{Link 1}
    \end{block}
\end{frame}

\begin{frame}
    \begin{block}{Consequences of the Inflation Reduction Act for Europe}
    US is subsidising everything, how should EU react?
    \end{block}
    \begin{block}{Financial Wind Contracts for Difference}
    Suggestion from Ingmar Schlecht, Lion Hirth and Christian Maurer to combine the benefits of contracts for difference (CfD) for developers (lower risk) and consumers (if prices are high) with market incentives of market premium system.\\
    \href{http://hdl.handle.net/10419/267597}{Link 1}
    \end{block}
\end{frame}

\begin{frame}
    \begin{block}{Wind-Ausschreibungen Rückblick}
    Seit 2017 erfolgt der politisch geförderte Ausbau von Windenergie an Land in Deutschland über regelmäßige Auktionen. Dazu legt die Bundesnetzagentur jeweils maximale Ausschreibungs-Mengen und zulässige Höchstpreise fest. Die Entwickler bieten die von ihnen zu installierenden Mengen und die gewünschte staatlich garantierte Mindestvergütung. In der Vergangenheit lagen die bezuschlagten Mengen oftmals unter den ausgeschriebenen Mengen. Vor diesem Hintergrund soll der Vortrag die Entwicklun ges Instruments diskutieren.\\
    \href{https://www.bundesnetzagentur.de/DE/Fachthemen/EektrizitaetundGas/Ausschreibungen/start.html}{Link 1}
    \href{https://www.wirtschaftsdienst.eu/inhalt/jahr/2023/heft/2/beitrag/gebotskostenfoerderung-in-windenergie-auktionen.html}{Link 2}
    \end{block}
    \begin{block}{Ausschreibungen von PV-Freiflächenanlagen}
    Ähnlich wie bei Windanlagen führt die Bundesnetzagentur auch regelmäßige Ausschreibungen von PV-Freiflächenanlagen durch. §§ 28 bis 35a und 37 bis 38b EEG liefern die gesetzliche Grundlage dafür. Mit der Einführung der PV-Ausschreibungen im Jahr 2017 wurden die von den Stromverbrauchern zu tragenden Belastungen des PV-Ausbaus deutlich gesenkt. D.er Vortrag soll die Details dieser Regelung sowie die Erfahrungen der letzten Jahre erörtern.\\
    \href{https://www.bundesnetzagentur.de/DE/Fachthemen/ElektrizitaetundGas/Ausschreibungen/Solaranlagen1/start.html}{Link 1}
    \end{block}
\end{frame}

\begin{frame}
    \begin{block}{Klimafreundlicher Luftverkehr - Überblick}
    Trotz steigender Ticketpreise und vermehrtem Verzicht auf Geschäftsreisen hat das Verkehrsvolumen der Luftfahrt fast wieder den Stand vor der Corona-Pandemie erreicht. Insbesondere für den Langstreckenverkehr dürfte die Batterie getriebene Elektromobilität vorerst aus Gewichtsgründen ausscheiden. Doch welche Möglichkeiten gibt es, um auch in diesem Sektor die Emission von Treibhausgasen drastisch zu senken? Forschungsinstitutionen, Flugzeugbauer und Luftfahrtgesellschaften arbeiten an einem Portfolio an Lösungen. Einen ersten Überblick verschafft \\
    \href{https://www.helmholtz-klima.de/aktuelles/klimafreundlich-fliegen}{Link 1}
    \end{block}
    \begin{block}{CO2-Schiff, CO2-Kreislauf - Tree Energy Solutions / Sequestrierung}
    Idea: Dual use ships arrive in Germany with LNG, take CO2 back to Norway for sequestration or to green-H2-rich countries for methanation (a la Tree Energy Solutions). What are the challenges, current status, prospects. \\
    \href{https://maritime-executive.com/article/study-dual-use-lng-shipping-could-transform-carbon-capture}{Link 1}
    \href{https://www.naturalgasworld.com/dual-use-shipping-an-idea-whose-time-has-yet-to-come-gas-in-transition-103245}{Link 2}
    \href{https://www.aiche.org/fscarbonmanagement/cmtc/2019/proceeding/paper/techno-economic-modeling-dual-purpose-lng-co2-shipping}{Link 3}
    \href{https://tes-h2.com/}{Link 4}
    \end{block}
\end{frame}

\begin{frame}
    \begin{block}{Industriestrompreis}
    Traditionell jammert die Industrie über zu hohe Strompreise, die ihre internationale Wettbewerbsfähigkeit beeinträchtigen würden. Schon in der Vergangenheit hat das Bundeswirtschaftsministerium nach Rücksprache mit der EU-Kommission in Einzelfällen Vergünstigungen gewährt. Als Spätfolge der Energiepreisexplosion im Jahr 2022 scheint jetzt aber ein globaler Subventionswettlauf eingesetzt zu haben. Anfang Mai 2023 hat das Bundeswirtschaftsministerium ein Arbeitspapier dazu vorgelegt.\\
    \href{https://www.bmwk.de/Redaktion/DE/Downloads/W/wettbewerbsfaehige-strompreise-fuer-die-energieintensiven-unternehmen-in-deutschland-und-europa-sicherstellen.html}{Link 1}
    \href{https://www.oeko.de/fileadmin/oekodoc/Stellungnahme-Strommarktdesign-Weiterentwicklung.pdf}{Link 2}
    \end{block}
\end{frame}
\end{document}
